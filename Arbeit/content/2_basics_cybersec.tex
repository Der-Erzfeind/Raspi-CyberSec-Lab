\chapter{Basics of Cybersecurity}\label{ch:basics_cybersec}
This chapter briefly explains the terms of IT security and cybersecurity and gives an overview of their goals and methods.
A summary of common threats in the cyberspace is given as well, giving context for \cref{ch:testing}.

\section{IT Security}
IT Security describes the protection of digital data and computer- and communication systems from unauthorized access.
Different principles are adhered to and methods and techniques applied in order to prevent the theft, interception, manipulation, and loss of data and systems. \cite{IT_Basiswissen}

\section{Cybersecurity}
IT- and cybersecurity are similar in that both have the same goals and apply the same principles and methods.
While IT Security focuses on classical computer systems and networks, cybersecurity focuses on data and systems in the cyberspace.
Included in the cyberspace are all systems and data connected to the internet.

\subsection{Goals}
The goals of cybersecurity include but are not limited to the well-known CIA triad, which is an acronym for Confidentiality, Integrity, and Availability \cite{Oriyano_2017}.
Confidentiality describes data only being accessible by the persons who are meant to do so.
Integrity describes data being unchanged after storage or transmission, meaning the data was not corrupted or manipulated.
Availability describes data and systems being accessible with the expected performance, whenever required. 

\subsection{Methods}\label{ssec:cs_methods}
Some methods for achieving the goals of cybersecurity are:

\begin{itemize}
    \item Security Awareness is the knowledge of the threats present in the cyberspace and methods for protection.
    "Awareness of the risks and available safeguards is the first line of defence for the security of information systems and networks" \cite[page~26]{IT_Basiswissen}.
    \item Encryption is a cryptographic\footnote{cryptography = science of secretive writing} method, which takes the readable data, called plaintext, and makes it unreadable for anyone besides the legitimate recipient.
    The encrypted text is called ciphertext and can be made readable again by decrypting it.
    To en- and decrypt a message, the sender and recipient have knowledge of a secret, often a password, which controls en-/decryption and is called key. \cite[page~1]{Watjen_2018}
    \item Authentication is the act of a user "performing some sort of action which proves that the claim a user is making about who they are is actually valid" \cite{Oriyano_2017}. 
    This is oftentimes done through the use of a password, in the case of Wi-Fi networks, the knowledge of the password proves that a user is allowed to access the network.
    \item Pentesting is a method to audit the security of a system or network. 
    It involves a security professional performing various types of attacks to discover weaknesses in soft- and hardware.
    \item Monitoring with anomaly detection can be used to detect the unauthorized access of systems and networks.
    In the network domain, so-called Intrusion Detection Systems (IDS) employ predictive models to recognize malicious activity inside a network \cite{geeksforgeeks_ids}.
    \item Incident Response is performed in case of a system breach in order to limit damage and secure the system.
\end{itemize}

\section{Threats and Risks in the Cyberspace}
There are various types of cybercriminals with different motives and intentions.
Motives of cybercriminals can vary, but commonly are financial, political, social, or military.
With these motives the intentions of an attack often are:
\begin{multicols}{2}
    \begin{itemize}
        \item blackmail
        \item espionage
        \item theft
        \item fraud
        \item sabotage
        \item corruption and insider deals
    \end{itemize}
\end{multicols}

The success rate and risk of an attack are in large parts determined by a target's attack area, which is the combination of the present vulnerabilities and attack vectors.
The term vulnerability describes a weak spot, which can be exploited by the attacker to break or circumvent security mechanisms.
An attack vector is the approach an attacker takes for carrying out an attack.
In regard to network security some common attack vectors are:

\begin{itemize}
    \item Denial of Service (DoS) attacks availability by preventing or limiting the functional capability of a system.
    \item Cracking describes the act of breaking or circumventing security mechanisms, e.g. reverse engineering passwords.
    \item Brute-Force is a technique for reverse engineering passwords by trail-and-error.
    \item Dictionary attack is a variation of Brute-Force, using a collection of leaked passwords to guess the correct solution.
    \item Spoofing refers to faking identities, achieved by the obfuscation of the own identity or the replication of another identity.
    \item Sniffing is the action of collecting all openly available data, e.g. recording unencrypted Wi-Fi frames.
    \item Man-in-the-Middle (MitM) attacks involve injection of a malicious device into a line of communication, which subsequently can manipulate and eavesdrop on the traffic.
    \item Social Engineering is a form of manipulation that abuses psychological mechanisms like trust, curiosity, or shock to exploit victims.
\end{itemize}

\cite[page~16-20]{IT_Basiswissen}


