\chapter{Summary}\label{ch:summary}
Increasing damages caused by cybercrime in recent years have shown the need for improving cybersecurity for companies, governments, and citizens.
The mismatch of demand and supply of trained security professionals represents a significant challenge in curbing the numbers of cybercrime predicted for the near future.
As a result, there is an urgent need for training students in the complex and multi-faceted discipline of cybersecurity.
Besides training professionals, raising awareness with employees and the public also plays an important role in protection against cybercrime.

The development of the Raspberry Pi CyberSec Lab aims to create a testing platform to aid in raising awareness and in education.

With the use of common components and a mostly modular software architecture, the RCSL lays the groundwork for a useful platform.
Testing showed that the RCSL is capable of providing environments for practicing network security, although certain flaws were revealed.
In summary, the goal of creating and testing a device, which combines usability with expandability, was mostly achieved.

Implementing the expansions and improvements proposed in \cref{ch:outlook}, the RCSL could become a useful asset in teaching cybersecurity.