\chapter{Cybersecurity Education}\label{ch:discussion}
In this chapter, the importance of cybersecurity education is highlighted, and the potential use cases for the RCSL in education are layed out.

\section{Importance of Cybersecurity Education}
The advancement of digital information technology, accelerated by trends like Artificial Intelligence, Cloud Computing or the Internet of Things (IoT), has reached a high pace, which companies and institutions are trying to keep up with.
In the pursuit of keeping up with this pace, convenience is often the main focus, whereas security and privacy are left as an afterthought. \cite[chapter~1]{Salmon_Levesque_McLafferty_2017}
This leads to the implementation and operation of vulnerable systems, presenting a larger attack surface for cybercriminals.

Experts estimate that cybercrime causes annual damages of 110 billion dollars worldwide and most companies, relying heavily on their IT infrastructure, can only operate for a few days without these systems \cite[page~3]{Hellmann_2023},\cite[page~12]{Salmon_Levesque_McLafferty_2017}. 
cyberattacks in the form of data piracy, or blackmailing have become commonplace and cause ever-increasing financial damages for companies, the public, and governments, as seen in the example depicted in \cref{fig:cost_of_cybercrime}.

\begin{figure}[h]
    \centering
    \includegraphics[width=0.8\textwidth]{figures/Abbildungen/statistic_id1399040_annual-cost-of-cybercrime-in-the-us-2017-2028.pdf}
    \caption{Estimated annual cost of cybercrime in the United States from 2017 to 2028 \cite{statista_cc}}
    \label{fig:cost_of_cybercrime}
\end{figure}

Cybersecurity awareness and education play a large role in ensuring the integrity of networks and systems. 
People without awareness and training in cybersecurity oftentimes tend to circumvent or ignore security mechanisms since following the principles of cybersecurity is often connected to effort \cite{Hellmann_2023}.
Awareness training should educate about the potential threads in the cyberspace and security measures to apply for mitigating the risk of falling victim to cybercrime.
Futhermore, the training should sensitize for common schemes of Social Engineering.
This helps to protect against threads like Social Engineering, Malware, or network attacks, such as MitM or Evil Twin. \cite[page~24-25]{IT_Basiswissen}, \cite{mariano2024wifi}

\section{Results Evaluation}
The general usability of the RCSL proved to be in line with the goals set for the project with intuitive and smooth operation.
The software architecture is mostly modular, allowing future expansion, and executes correctly implemented functions reliably.
For improved development and problem analysis as well as performance, a rewrite of the code with adherence to software engineering best practices would be adviceable.

Concluding the tests, it appears the RCSL can provide a relatively reliable environment to perform network attacks in.
However, the generation of artificial network traffic could aid the speed of cracking WEP, as well as depict a more realistic scenario, and might be considered for future improvements.
The error caused by wpa\_supplicant, described in \cref{sec:network_management}, seems to have been fixed, but should also be revisited in order to implement a more dependable fix.


\section{Potential Use Cases}
Confrontation with realistic scenarios and hands-on practice are effective tools for raising awareness and teaching cybersecurity.
Authentic experiences help to "reinforc[e] the importance of [security]" \cite{mariano2024wifi} and connect previously learned theory to the experience in real world application. 
Providing context for learning is essential, as it helps to build mental schemas and understanding of the core concepts, enhancing the ability to transfer the knowledge to different scenarios. \cite[page~61]{Cybersec_Edu}

The RCSL is able to create test environments for various applications and could be used as tool in education for demonstration purposes and to allow students to get hands-on experience.

A lecturer could use the device to demonstrate the process of cracking a Wi-Fi network and performing subsequent attacks like MitM to eavesdrop on the traffic.
The MQTT communication between the RPI and ESP32 could be a suitable target for such an attack.
Demonstration of the ease with which an improperly secured network can be cracked and data stolen or manipulated would most likely give a vivid impression of the threats present in the cyberspace.
This memorable experience intern might lead to the perception of the importance of cybersecurity being increased, aiding the spread of awareness.

For cybersecurity students, the RCSL could serve as a platform to practice pentesting techniques on, providing context for theoretical concepts and practical experience.
The attacks performed in \cref{ch:testing} are an obvious choice as well as various network attacks, such as MitM, DoS, or TLS/SSL attacks.
Besides pentesting, future expansions of the project could also include the implementation of protective measures, such as IDS systems, allowing students to gain experience in their application.

The Juice Shop could be employed for teaching security practices in software engineering and web development, providing context for the adherence to principles of secure development.

With the future addition of features, like the ones suggested in \cref{ch:outlook}, the RCSL could be expanded to cater for numerous further use cases.