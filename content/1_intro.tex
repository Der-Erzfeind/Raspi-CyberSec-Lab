\chapter{Introduction}\label{ch:intro}
The modern world is moving towards digitalization at a high pace, which aims to make it smart and interconnected. 
This transformation comes with the ever-increasing use of connected devices and services, which oftentimes carry sensitive data and are not always protected sufficiently.
As a result, there is a rising number of cybercrime incidents and subsequent damages estimated to more than \$1 trillion worldwide.

Compounding on top of the situation is a mismatch between demand and supply of cybersecurity professionals and a public, which is in large parts not aware of the risks posed by the use of connected IT technology.
Cybercriminals can often easily exploit these unaware people to gain access to restricted systems and data.
%People who are unaware of potential risks are often a weak spot and can be exploited by cyber criminals to access systems and networks. 
To combat this problem, there's a need for training cybersecurity professionals, as well as to raise awareness at every level of education and the public.
As a method for raising awareness, teaching principles of cybersecurity and the methods to apply them, live demonstration and hands-on learning with realistic risk scenarios can be an effective tool. \cite{mariano2024wifi}\cite{Cybersec_Edu}


\section{Goal}
The goal of this project is to develop a platform with a flexible and expandable architecture, which can be used for demonstration and practice of Pentesting\footnote{penetration testing, see \cref{ssec:cs_methods}} in cybersecurity education.
A Raspberry Pi will be used as the main device to create a platform that is portable and easy to use.
It should be able to reliably set up testing environments for various fields of application, to allow a lecturer to demonstrate cyberattacks and students to practice penetration testing in a realistic environment.
In recent years, trends like the Internet of Things (IoT) have produced many headless\footnote{without any means for direct interaction} devices, which are often linked to smartphones for user interaction, using Wi-Fi or Bluetooth for communication.
For this reason, wireless communication security in the form of Wi-Fi technology will be implemented as the first application specific testing environment.
To teach secure design of web applications, the device should be able to host a server with the OWASP Juice Shop\footnote{is a web security training program, see \cref{ssec:juice_shop}}, which again ties into the security of IoT devices, because they are oftentimes accessed via web interfaces, which are often the source of vulnerabilities \cite[page~174]{Hellmann_2023}.

\section{Overview}
\begin{itemize}
    \item \cref{ch:basics_cybersec} explains the difference between IT security and cybersecurity and outlines the principles and encountered threats.
    \item \cref{ch:basics_wifi} explores the history of Wi-Fi technology and explains its architecture, function and security mechanisms.
    \item \cref{ch:documentation} documents the hardware and software of the RCSL.
    \item \cref{ch:development} outlines the development process and the challenges faced along the process.
    \item \cref{ch:testing} explains the penetration tests conducted and documents the results of testing them on the RCSL.
    \item \cref{ch:discussion} outlines the challenges of cybersecurity education and proposes use cases for the RCSL.
\end{itemize}