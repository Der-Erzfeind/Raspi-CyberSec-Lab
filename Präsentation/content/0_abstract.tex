\thispagestyle{empty}

\section*{Abstract}\label{sec:abstract}
This thesis documents the development and testing of the Raspberry Pi Cybersecurity Laboratory (RCSL).
The RCSL is intended as a testing platform for demonstration and practice purposes of cybersecurity in various IT applications.

An expandable software architecture is developed, which focuses on usability and includes features for the creation of test environments in the Wi-Fi and web application domain.
Development and functionality are documented, highlighting the challenges encountered along the process.

The RCSL's Wi-Fi functionality is tested by performing attacks on the WEP and WPA/WPA2 network standards.
Several faults in the system were discovered, which prohibited the execution of the tests.
After elimination of the errors, cracking of both standards was successfully performed multiple times.
The results show the RCSL to be capable of establishing testing environments for Wi-Fi security.

Proposed usecases for the RCSL include demonstration and hands-on practice in cybersecurity and software development education.
Future improvement and expansion of the project are recommended.


\newpage
\thispagestyle{empty}
\mbox{}  % Empty content for the first blank page
\newpage

\section*{Kurzdarstellung}
\label{sec:kurzdarstellung}
Diese Arbeit dokumentiert die Entwicklung und Erprobung des Raspberry Pi Cyber Security Laboratory (RCSL).
Das RCSL ist als Testplattform für Demonstrations- und Übungszwecke von Cybersecurity in verschiedenen IT-Anwendungen gedacht.

Es wird eine erweiterbare Software-Architektur entwickelt, die sich auf die Benutzerfreundlichkeit konzentriert und Funktionen für die Erstellung von Testumgebungen im Wi-Fi- und Web-Anwendungsbereich enthält.
Entwicklung und Funktionalität werden dokumentiert und die dabei aufgetretenen Herausforderungen aufgezeigt.

Die Wi-Fi-Funktionalität des RCSL wird durch Angriffe auf die Netzwerkstandards WEP und WPA/WPA2 getestet.
Es wurden mehrere Fehler im System entdeckt, die die Durchführung der Tests verhinderten.
Nach Beseitigung der Fehler wurden beide Standards mehrfach erfolgreich geknackt.
Die Ergebnisse zeigen, dass das RCSL in der Lage ist, Testumgebungen für die Wi-Fi-Sicherheit einzurichten.

Vorgeschlagene Anwendungsfälle für den RCSL sind Demonstrationen und praktische Übungen in der Cybersecurity- und Softwareentwicklungslehre.
Zukünftige Verbesserungen und Erweiterungen des Projekts werden empfohlen.